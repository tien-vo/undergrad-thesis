\subsection{Particle advance}
The Hamiltonian equation of motion in \cref{eq:hamiltonian_eom}, although useful
for analysis, is only an approximation near a single resonance. \cite{Roth1999}
alternated between that and the exact Lorentz force to reduce the computational cost for particles entering resonance, since adaptive RK of the 4th order is
expensive. However, in doing so, the code user must impose an arbitrary boundary
in switching between the resonant and non-resonant regimes. Here, we shall use
the relativistic Boris pusher from \cite{Ripperda2018} to solve for the full
Lorentz force and rely on its volume-preserving characteristics to choose the
appropriate step size. However, we must first describe our normalizations. From
\cref{sec:whistler}, it is natural to normalize $\vb{B}\to\vb{B}/B_0$ and
subsequently $\vb{E}\to \vb{E}/cB_0$. Since we are using relativistic
formulations, $\vb{v}\to\vb{v}/c$ and $\vb{P}\to\vb{P}/mc$. The characteristic
frequency in our system is defined by the electron cyclotron frequency
$\Omega_{ce}$, so the wave frequency $\omega\to\omega/\Omega_{ce}$ and time
$t\to t\Omega_{ce}$. The spatial position thus becomes $\vb{r}\to\vb{r}\Omega_{ce}/c$.

The description of the Boris algorithm is as follows. The Lorentz force in
natural units has the form $d\vb{u}/dt=s\qty(\vb{E}+\vb{v}\times\vb{B})$ where
$\vb{u}=\gamma\vb{v}$ and $\gamma=\sqrt{1+u^2}$. The time-centered finite difference expression of this is
\begin{equation}
 \vb{u}_{n+1}-\vb{u}_n=s\Delta
t\qty[\vb{E}_n+\qty(1/2\gamma_n)\qty(\vb{u}_{n+1}+\vb{u}_n)\times\vb{B}_n]
\end{equation}
 where $\vb{u}_n=\gamma_n\vb{v}\qty(t_n-\Delta t/2)$, $\vb{E}_n=\vb{E}_n(t_n)$,
$\vb{B}_n=\vb{B}_n\qty(t_n)$ and $\Delta t$ is the step size where
$t_n=n\Delta t$ for $n\in\N$. $\gamma_n$ is the Lorentz factor determined from
$u_n$. Now, the Kick-Drift-Kick steps that make this algorithm a leapfrog scheme
are defined via the two auxilliary vectors $\vb{u}_\pm$. The first kick is a
half electric field acceleration from $\vb{u}_n$ to $\vb{u}_-=\vb{u}_n+(s\Delta
t/2)\vb{E}_n$, followed by a rotation $\vb{u}_-\to\vb{u}_+$ by the magnetic
field $\vb{u}_+=\vb{u}_-+\qty(\Delta
t/2\gamma_n)(\vb{u}_++\vb{u}_-)\times\vb{B}_n$. $\vb{u}_+$ here seems to be
implicitly defined, but from the geometry of this rotation, it can be computed
explicity as $\vb{u}_+=\vb{u}_-+\qty(\vb{u}_-+\vb{u}_-\times\vb{T})\times\vb{S}$
with $\vb{T}=\qty(s\Delta t/2\gamma_n)\vb{B}_n$ and $\vb{S}=2\vb{T}/(1+T^2)$
(see more details in \cite{Birdsall&Langdon1985}). Then the second kick accelerates the particle to the next state $\vb{u}_{n+1}=\vb{u}_++\qty(s\Delta
t/2)\vb{E}_n$.

To simulate a single uniform whistler fields in natural units,
\cref{eq:whistler_EM_fields} can be rewritten as
\begin{equation}\label{eq:simulate_B}
    \frac{\vb{B}_w}{B_0}=\frac{E_x^w}{cB_0}\qty[\qty(\frac{cB_x^w}{E_x^w})\sin\psi\xhat+\qty(\frac{cB_y^w}{E_x^w})\cos\psi\yhat+\qty(\frac{cB_z^w}{E_x^w})\sin\psi\zhat]
\end{equation}
and similarly,
\begin{equation}\label{eq:simulate_E}
    \frac{\vb{E}_w}{cB_0}=\frac{E_x^w}{cB_0}\qty[\cos\psi\xhat-\qty(\frac{E_y^w}{E_x^w})\sin\psi\yhat+\qty(\frac{E_z^w}{E_x^w})\cos\psi\zhat]
\end{equation}
Since the STEREO spacecraft only measured the whistler electric field amplitudes
\citep{Breneman2010}, $E_x^w$ is used as
the scaling factor. The unitless polarizations can be computed with
\cref{eq:whistler_polarizations}. Note that the wave phase in natural units
is $\psi=\omega\qty(N_\perp x+N_\|z-t)$, and that it is zero for particles
starting out at the origin at $t=0$. Originally, the wave has an amplitude
$E_{w}^0=E_x^w\sqrt{1+\qty(E_z^w/E_x^w)^2}$. So, $E_x^w$ will be chosen such
that $E_{w}^0$ has a desired physical value. To simulate a wave packet with
the same original wave amplitude $E_{w}^0$ and $N$ frequencies
$\omega_j=\omega_1+(j-1)\Delta\omega$ with spacing $\Delta\omega$, the
calculations \cref{eq:simulate_B} and \cref{eq:simulate_E} can simply be
repeated and the total fields are $\vb{E}_w=\sum_{j=1}^N\vb{E}_{w,j}$ and
$\vb{B}_w=\sum_{j=1}^N\vb{B}_{w,j}$.

With these calculations, a description of the particle advance at each time step
is completed. The scaling factor is calculated at the beginning of the
simulation. So each loop involves (a) calculating new wave phase, constructing
the total field, and advancing the particle, (b) the diagnostics, and (c)
writing to database. (b) and (c) can be activated at different time intervals.

